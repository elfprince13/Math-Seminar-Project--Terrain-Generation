 
\documentclass{article}

    \usepackage{fullpage}          %This will give you the standard 1in. margins, 12pt font, single spaced, page numbers, ect.  
    \usepackage{url}                 %This is intended if you have a website in the bibliography
    \usepackage{graphicx}         %This is if you want to include a picture in your document


    \usepackage{amsmath, amsthm, amssymb}
    \usepackage{epsfig}
    \usepackage{pslatex}
     

    \title{Techniques for Fractal Terrain Generation}
    \author{Krista Bird \and Thomas Dickerson \and Jessica George}
    \begin{document}
    \maketitle

     \begin{abstract}
     In this paper we present various methods of fractal terrain generation. We discuss the relative advantages and disadvantages of each method. Methods presented include: midpoint displacement, Fourier space manipulation of gaussian noise, multi fractal methods, and their respective variations. 
     \end{abstract}

	\section{Midpoint Displacement Method}
		Generating fractal terrains by midpoint displacement is a relatively straightforward method for generating artificial terrain with realistic features. The dimensions of the terrain must be a square with sides of length $2^{N}+1$, so that each sub-square has an exact middle lying on a gridpoint. Given a heightfield,
		\begin{equation}
		T = \left[ \begin{matrix}
		T_{0,0} & \ldots & T_{0,2^{N},} \\
		\vdots  &  \ddots & \vdots \\
		T_{2^{N},0} & \ldots & T_{2^{N},2^{N}}
		\end{matrix} \right]
		\end{equation}
then,
		\begin{equation}
			T_{k \times 2^{n},0} = \frac{T_{(k+1) \times 2^{n},0} + T_{(k-1) \times 2^{n},0}}{2} + A \times R \times 2^{-H \times n}
		\end{equation}
where
		\begin{equation}
			k \equiv (2 \times l)+1 \pmod{2^{N-n}}
		\end{equation}
for $n < N$ and $l \in \mathbb{Z}$.
	\subsection{Variations: Diamands and Squares}
		One criticism of the realism of the terrain generated by Midpoint Displacement is that it sometimes leaves square-shaped artifacts in the terrain. The Diamonds and Squares method attempts to alleviate this by alternating calculated values to square and diamond patterned midpoints. 
	\subsection{Implementation Challenges}
		Maple has a very small limit on stack space for use by recursively defined functions, so the algorithm must be rewritten as an iterative one.
		Additionally, Maple uses 1-indexed arrays, which makes a few formulas slightly bulkier.

	\section{Using the Fast Fourier Transform to Generate Fractal Terrain} 
	\label{sec:FFT}
	A Fourier transform is another method to generate fractal terrains. This is generally used to generate older terrains with a flatter landscape. The landscape created by a Fourier transform can be tiled. The Fourier transform enables us to express random noise as a function of sine and cosine functions, and it converts the function to a frequency domain. The result is a fractal landscape with smooth rolling features rather than ridges and peaks that can be generated using other methods. Thus, the landscape is a sum of the sine and cosine waves at different frequencies.
	
	Unlike many other processes for fractal terrain generation, the Fourier method is not an iterative process (Stanger). Initially, we begin using a random Gaussian noise (other types of noise can be used here as well), which is a two dimensional grid of discrete random values. Using Maple, we are able to apply the fast Fourier transform (FFT), which preforms a discrete Fourier transform (DFT). The FFT allows us to decompose the random noise into the sum of the sine and cosine functions. In Fourier space, the noise is broken into cells. Within each cell, we can find the magnitude of the wave with that frequency. These magnitudes are then be converted to the frequency domain. In this domain, each frequency is a complex number. This is all done through the use of the FFT algorithm.
	
	Once the values are in the frequency domain, we must scale these frequencies because we will have some that are very high and some that are very low. We do this using a frequency filter of the form $1/f*r$. The $f$ is the frequency represented by that value and the $r$ is a roughness parameter. After we have scaled the frequencies we can apply an inverse fast Fourier transform in order to generate a fractal landscape that is the sum of the waves at different frequencies. This is where the landscape gets its get the "rolling" quality from (Kareem).We apply this FFT to a two diminutional random noise, but we are able to view the data in three dimensions.The result is a fractal landscape. There are no artificial ridges or peaks and the landscape is easily tiled.
	
	The characteristics of the landscape are determined by the random seed number. How rough or smooth the surface of the landscape is depends on the value you scale the frequencies by. The higher the number you scale by, the smoother the landscape. The smaller the scaling number, the rougher the landscape.
	    blah blah blah~\cite{Jo09, PaperII}.
     
    \bibliography{terrainbib}
    \bibliographystyle{plain}

     

    \end{document}

